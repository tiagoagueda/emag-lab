%
%

\question
\textbf{Trabalho Pr\'{a}tico 5 : Bobinas de Helmholtz}

Recorde o trabalho pr\'{a}tico 5, sobre o estudo do campo magn\'{e}tico produzido pelas bobinas de Helmholtz.
\begin{parts}
\part
Este estudo foi feito com base numa sonda de efeito de Hall. Justifique a necessidade de proceder \`{a} calibra\c{c}\~{a}o da sonda de campo magn\'{e}tico.

\part
A calibra\c{c}\~{a}o foi feita recorrendo a um solenoide padr\~{a}o com $3467\pm60$ espiras por metro. As medidas obtidas est\~{a}o presentes na Tabela~\ref{tab:calibhall}.

\begin{table}[h]
\caption{\label{tab:calibhall}Tens\~{a}o ($V_H$) amplificada da sonda por efeito de Hall em fun\c{c}\~{a}o da corrente ($I_S$) que atravessa o solenoide padr\~{a}o.}
\begin{center}
\begin{tabular}{|c|c|}
\hline
$I_S~\pm~0.01$ (A) & $V_H~\pm~0.1$ (mV) \\ \hline
0.08 & 32.5 \\ 
0.18 & 48.5 \\ 
0.29 & 64.5 \\ 
0.47 & 91.9 \\ 
0.69 & 126.4 \\ 
0.82 & 145.8 \\ \hline
\end{tabular}
\end{center}
\end{table}

Com base nestes valores, represente graficamente a tens\~{a}o de Hall ($V_H$) em fun\c{c}\~{a}o da corrente do solenoide ($I_S$).

\part 
Obtenha a constante de calibra\c{c}\~{a}o da sonda de efeito de Hall, bem como o respetivo erro.

Nota: O campo magn\'{e}tico ao longo do eixo de um solenoide infinito \'{e} dado pela Eq.~\ref{eq:solenoide}.
\begin{equation}
\label{eq:solenoide}
\left|\overrightarrow{B_S}\right|=\mu_0\frac{N}{l}I_S
\end{equation}

\part
Para \underline{uma bobina} com $N$~espiras colocou-se a sonda de efeito de Hall no centro da mesma tendo sido medido o valor de $V_H=50~\text{mV}$. Sabendo que a corrente que atravessa a bobina \'{e} $I=0.50~\text{A}$, o raio da bobina \'{e} de $R=3.5~\text{cm}$ e que a norma do campo magn\'{e}tico no eixo de uma espira ao longo do seu eixo \'{e} dado pela Eq.~\ref{eq:anel}, determine o n\'{u}mero de espiras da bobina, $N$.
\begin{equation}
\label{eq:anel}
B\left(x\right)=\frac{\mu_0}{2}\frac{IR^2}{\left(R^2+x^2\right)^{3/2}}
\end{equation}
\end{parts}
