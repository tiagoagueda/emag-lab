\question[25]

\textbf{Circuitos de Corrente Cont\'{i}nua}

Durante a realiza\c{c}\~{a}o do trabalho pr\'{a}tico 3 foram apresentadas duas configura\c{c}\~{o}es destintas de medi\c{c}\~{a}o de uma resit\^{e}ncia desconhecida $R_x$, a partir da medi\c{c}\~{a}o, atrav\'{e}s de um amper\'{i}metro, da corrente $I$ que a atravessa e da medi\c{c}\~{a}o da queda de tens\~{a}o nesta, atrav\'{e}s de um voltimetro, conforme ilustra a Fig.~\ref{fig:medicaorx}.

\begin{figure}[h]
\centering
\begin{tabular}{cc}
\begin{pspicture}[showgrid=false](5,4)
\pnode(0,0.5){A}
\pnode(0,2){B}
\pnode(2.5,2){C}
\pnode(5,2){D}
\pnode(5,0.5){E}
\pnode(0,3.5){F}
\pnode(5,3.5){G}
\vdc(A)(E){}
\resistor[dipolestyle=zigzag](C)(B){$R_x$}
\circledipole[labeloffset=0](C)(D){\Large\textbf{A}}
\circledipole[labeloffset=0](F)(G){\Large\textbf{V}}
\wire(A)(B)
\wire(D)(E)
\wire(B)(F)
\wire(D)(G)
\end{pspicture}&
\begin{pspicture}[showgrid=false](5,4)
\pnode(0,0.5){A}
\pnode(0,2){B}
\pnode(2.5,2){C}
\pnode(5,2){D}
\pnode(5,0.5){E}
\pnode(0,3.5){F}
\pnode(2.5,3.5){G}
\vdc(A)(E){}
\resistor[dipolestyle=zigzag](C)(B){$R_x$}
\circledipole[labeloffset=0](C)(D){\Large\textbf{A}}
\circledipole[labeloffset=0](F)(G){\Large\textbf{V}}
\wire(A)(B)
\wire(D)(E)
\wire(B)(F)
\wire(C)(G)
\end{pspicture}\\
a)~Longa deriva\c{c}\~{a}o & b)~Curta deriva\c{c}\~{a}o
\end{tabular}
\caption{\label{fig:medicaorx}Configura\c{c}\~{o}es possiveis na medi\c{c}\~{a}o de $R_x$.}
\end{figure}

Sabendo que $R_A$ e $R_V$ s\~{a}o as resist\^{e}ncias internas do amperimetro e volt\'{i}metro, respetivamente, se pretender-mos medir a resist\^{e}ncia $R_x$, sabendo \`{a} partida que $R_x \gg R_A$, qual das duas configura\c{c}\~{o}es anteriores deveremos usar? \underline{Justifique convenientemente a sua resposta}.

\begin{solution}
Treta2!
\end{solution}