\question
\textbf{Trabalho Pr\'{a}tico 6 : Indu\c{c}\~{a}o Eletromagn\'{e}tica}

Recorde o trabalho pr\'{a}tico sobre a Indu\c{c}\~{a}o eletromagn\'{e}tica.

\begin{parts}
\part[10]
Na primeira parte deste trabalho montou o circuito da Fig.~\ref{fig:bobinas}, com o objetivo de estudar a Lei de Faraday e de Lenz, na qual se varia o fluxo magn\'{e}tico atrav\'{e}s de uma bobina usando um \'{i}man.

\begin{subparts}
\subpart
Descreva o que observou no movimento dos ponteiros do galvan\'{o}metro, quando aproximou o \'{i}man \`{a} bobina e, \`{a} luz das leis de Faraday e de Lenz, indique o sentido da corrente induzida no circuito da bobina.

\subpart E quando afastou o \'{i}man da bobina?
\end{subparts}

\begin{figure}[h]
\centering
\begin{pspicture}[showgrid=false](8,4)
\pnodes(1,1){A}(1,3){B}(7,3){C}(7,1){D}

\wire(A)(B)
\coil[dipolestyle=curved,
      labeloffset=-.5](B)(C){$1200~\text{esp.}$}
\pspolygon[fillcolor=red,fillstyle=solid](1.2,2.8)(2,2.8)(2,3.2)(1.2,3.2)
\rput*[t]{0}(1.6,2.6){S}
\pspolygon[fillcolor=blue,fillstyle=solid](2,2.8)(2.8,2.8)(2.8,3.2)(2,3.2)
\rput*[t]{0}(2.4,2.6){N}
\psline[linewidth=2pt]{<->}(1.5,3.5)(2.5,3.5)
\wire(C)(D)
\circledipole[labeloffset=0](A)(D){\Large\textbf{G}}

\end{pspicture}
\caption{\label{fig:bobinas}}
\end{figure}

\part\label{part:m}
Considere um circuito no qual est\~{a}o ligados em s\'{e}rie um gerador de sinal, uma resist\^{e}ncia de $R=100~\Omega$ e um solenoide de raio $r=1.38~\text{cm}$ e $N/l=3000$ espiras por metro. Uma bobina de $N_b=1200$ espiras envolve o solenoide.
\begin{subparts}
\subpart[20]
Sabendo que a for\c{c}a eletromotriz (f.e.m.) induzida na bobina dada pela Eq.~\ref{eq:fem}, determine o coeficiente de indu\c{c}\~{a}o m\'{u}tua esperado.

\begin{equation}
\label{eq:fem}
\varepsilon=\pi\mu_0\frac{N}{l}r^2N_b\frac{\mathrm d i}{\mathrm d t}
\end{equation}

\end{subparts}

\part 
Considere um circuito no qual est\~{a}o ligados em s\'{e}rie um gerador de sinal e uma bobina ($1200~\text{espiras}$, resist\^{e}ncia de $13~\Omega$ e coeficiente de auto-indut\^{a}ncia $L=54~\text{mH}$.)

Aplicou-se um sinal sinusoidal \`{a} bobina, observando-o no Canal A do oscilosc\'{o}pio (Fig~\ref{fig:osci}. - \textit{tracejado}) tendo-se observado no Canal B dos oscilosc\'{o}pio (Fig~\ref{fig:osci}. - \textit{cheio}) o sinal induzido no solenoide.

\begin{figure}[h]
\begin{center}
\newpsstyle{Dash}{linestyle=dashed,
linecolor=black,linewidth=0.035,plotpoints=50}
\psscalebox{1}{
\Oscillo[Wave1=\SinusA,
         Wave2=\SinusB,
    amplitude1=15.5,
    amplitude2=1.2,
       period1=.44,
       period2=.44,
        phase1=0.0,
        phase2=0.0,        
    sensivity1=5.0,
    sensivity2=1.0,
       timediv=0.1,       
    plotstyle1=Dash,
      AllColor=false]}
\caption{\label{fig:osci}Visor do oscilosc\'{o}pio. O Canal A est\'{a} representado a tracejado enquanto o Canal B a cheio.}
\end{center}
\end{figure}

\begin{subparts}
\subpart[25]
Calcule o coeficiente de indu\c{c}\~{a}o m\'{u}tua do circuito e compare-o com o valor obtido em~(\ref{part:m}), 

\subpart[20]
Se o sinal induzido (Canal B) fosse praticamente constante no tempo, o sinal aplicado (Canal A) seria triangular? \underline{Justifique} a sua resposta.

\end{subparts}
\end{parts}