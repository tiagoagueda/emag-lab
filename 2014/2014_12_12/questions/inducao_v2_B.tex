\question
\textbf{Trabalho Pr\'{a}tico 6 : Indu\c{c}\~{a}o Eletromagn\'{e}tica}

Recorde o trabalho pr\'{a}tico sobre a Indu\c{c}\~{a}o eletromagn\'{e}tica.

\begin{parts}
\part[10]
Na primeira parte deste trabalho montou o circuito da Fig.~\ref{fig:bobinas}, com o objetivo de estudar a Lei de Faraday e de Lenz.

Descreva o que observou no movimento dos ponteiros do galvan\'{o}metro quando, ap\'{o}s ter mantido o circuito ligado, desligou a fonte de tens\~{a}o e, \`{a} luz das leis de Faraday e de Lenz, indique o sentido da corrente induzida nesse instante (em que desligou) no circuito da bobina 2.

\begin{figure}[h]
\centering
\begin{pspicture}[showgrid=false](9,5)
\pnodes(1,1){A}(1,4){B}(4,4){C}(4,1){D}(5,1){E}(5,4){F}(8,4){G}(8,1){H}
\vdc[labeloffset=1](A)(D){$15~\text{V}$}
\resistor[dipolestyle=zigzag,labelangle=:U](A)(B){$5~\Omega$}
\coil[dipolestyle=curved,
      labelangle=:U,
      intensitylabel=$I$,
      labeloffset=-.5](C)(D){$1200~\text{esp.}$}
\newSwitch[tensionlabel=Bobina 1,
           tensioncolor=white,
           ison=false,
           tensionlabeloffset=0.7](B)(C){}

\coil[dipolestyle=curved,
      labelangle=:U,
      labeloffset=-.5](E)(F){$3600~\text{esp.}$}
\wire(G)(H)      
\circledipole[tensionlabel=Bobina 2,
           tensioncolor=white,
           tensionlabeloffset=0.7,
           labeloffset=0.0](F)(G){\Large\textbf{G}}
\wire(E)(H)

\end{pspicture}
\caption{\label{fig:bobinas}}
\end{figure}

\part\label{part:m}
Considere um circuito no qual est\~{a}o ligados em s\'{e}rie um gerador de sinal, uma resist\^{e}ncia de $R\pm\Delta R=100\pm5~\Omega$ e um solenoide (daqui em diante referido como \emph{prim\'{a}rio}) de raio $r\pm\Delta r=1.38\pm0.01~\text{cm}$ e $N/l\pm\Delta N/l=3000\pm15$ espiras por metro. Uma bobina (daqui em diante referida como \emph{secund\'{a}rio}) de $N_b\pm\Delta N_b=1200\pm6$ espiras envolve o \emph{prim\'{a}rio}. Aplica-se um sinal triangular ao circuito \emph{prim\'{a}rio} tendo-se observado nos terminais da resist\^{e}ncia o sinal do Canal A da Fig.~\ref{fig:osci} (\emph{tracejado}). O sinal induzido no \emph{secund\'{a}rio} \'{e} o que se observa no Canal B da mesma figura (\emph{cheio}).

\begin{subparts}
\subpart[30]
Determine o coeficiente de indu\c{c}\~{a}o m\'{u}tua observado e compare-o com o esperado, sabendo que a for\c{c}a eletromotriz (f.e.m.) induzida no \emph{secund\'{a}rio} \'{e} dada pela Eq.~\ref{eq:fem}.

\begin{equation}
\label{eq:fem}
\varepsilon=\pi\mu_0\frac{N}{l}r^2N_b\frac{\mathrm d i}{\mathrm d t}
\end{equation}

\begin{figure}[h]
\begin{center}
\newpsstyle{Dash}{linestyle=dashed,
linecolor=black,linewidth=0.035,plotpoints=50}
\psscalebox{1}{
\Oscillo[Wave1=\TriangleA,
         Wave2=\RectangleB,
    amplitude1=4.00,
    amplitude2=-0.3,
       period1=1.2,
       period2=1.2,
        phase1=0.0,
        phase2=0.0,        
    sensivity1=1.0,
    sensivity2=0.2,
       timediv=0.15,
       offset2=+0.9,        
    plotstyle1=Dash,
      AllColor=false]}
\caption{\label{fig:osci}Visor do oscilosc\'{o}pio. O Canal A est\'{a} representado a tracejado enquanto o Canal B a cheio.}
\end{center}
\end{figure}

\subpart[15]
Justique convenientemente a forma do sinal induzido (Fig.~\ref{fig:osci}~-~\textit{cheio}).

\subpart[10]
Porque \'{e} que o sinal induzido (Fig.~\ref{fig:osci}~-~\textit{cheio}) apresenta flutua\c{c}\~{o}es nas extremidades?

\subpart[10]
Se o sinal aplicado (Canal A) fosse do tipo "dente-de-serra", o sinal induzido (Canal B) seria triangular?

\end{subparts}
\end{parts}