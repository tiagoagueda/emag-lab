%
%
\question[30]
\textbf{Eletroest\'{a}tica}

Um aluno eletriza uma barra de PVC e aproxima-a da cabe\c{c}a do eletrosc\'{o}pio de folha, at\'{e} finalmente a tocar com a ponta da barra. Explique o que acontece ao longo do processo experimental, explicitando todos os fluxos de carga relevantes. Diga tamb\'{e}m, justificando, se a experi\^{e}ncia lhe permite determinar o sinal da carga na barra.

\begin{solution}

Quando o aluno eletriza por fric\c{c}\~{a}o uma barra de PVC esta fica carregada negativamente na zona em que foi friccionada. O pano usado fica por sua vez carregado com carga oposta.

Inicialmente o eletrosc\'{o}pio encontra-se descarregado. Ao aproximar a barra da cabe\c{c}a do eletrosc\'{o}pio, por indu\c{c}\~{a}o eletrost\'{a}tica, esta fica carregada positivamente. Como h\'{a} um fluxo de cargas positivas para a cabe\c{c}a do eletrosc\'{o}pio, as palhetas (na outra extremidade) ficam com excesso de cargas negativas.

\begin{center}
\begin{pspicture}[showgrid=true](6,6)

\rput*[t]{20}(2.6,2.5){\psline(0,0)(0,-2)}
\pspolygon[fillstyle=solid,fillcolor=lightgray]
 (3.5,4)(1.5,4)(1.5,3.5)(2.25,3.5)(2.25,3)(2.75,3)(2.75,3.5)(3.5,3.5)
\pspolygon[fillstyle=solid,fillcolor=lightgray]
 (2.4,3)(2.4,0.5)(2.6,2.5)(2.6,3)
 
\psline(1,0)(4,0)(4,3)(1,3)(1,0)
\end{pspicture}
\end{center}

Como as duas palhetas finas e de massa muito pequena ficam ambas carregadas com a mesma carga, devido \`{a} for\c{c}a eletrost\'{a}tica repulsiva entre as cargas de mesmo sinal, as palhetas afastam-se uma da outra (Fig.~). Este fen\'{o}meno ocorre, independentemente da carga induzida, logo \'{e} imposs\'{i}vel avaliar o sinal da carga induzida na cabe\c{c}a do eletrosc\'{o}pio.

Ao tocar com a barra de PVC na cabe\c{c}a do eletrosc\'{o}pio, como s\~{a}o de material mau condutor, as cargas presentes na cabe\c{c}a n\~{a}o ir\~{a}o fluir totalmente para a barra. Ao afastar-se a barra da cabe\c{c}a, o eletrosc\'{o}pio tende para o equil\'{i}brio eletrost\'{a}tico. Se parte da carga tiver sido descarregada pela barra aquando do contacto, uma carga residual permanece nas palhetas e elas permanecem afastadas.
\end{solution}