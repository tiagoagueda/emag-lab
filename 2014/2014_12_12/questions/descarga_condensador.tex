%
%

\question
\textbf{Trabalho Pr\'{a}tico 4 : Circuito RC}

Recorde o Trabalho Pr\'{a}tico 4, no qual estudou a resposta transit\'{o}ria do circuito RC.

Neste trabalho procedeu ao descarregamento de um condensador, com capacidade conhecida $C=10~000\pm 20\%~\mu\text{F}$, usando o circuito da Fig.~\ref{fig:descargcond} e registou na Tabela~\ref{tab:vcondensador} a d.d.p. nos terminais do condensador em fun\c{c}\~{a}o do tempo $v_D\left(t\right)$.

\begin{figure}[h]
\centering
\begin{pspicture}[showgrid=false](11,5)
\pnodes(1,1){A}(1,4){B}(3,4){C}(6,4){D}(6,1){E}(9,4){F}(9,1){G}
\vdc[labeloffset=1.1](B)(A){$15~\text{V}$}
\newSwitch[ison=false](B)(C){$S_1$}
\newSwitch[ison=true](D)(F){$S_2$}
\resistor[dipolestyle=zigzag](C)(D){$R_1=1k5$}
\resistor[dipolestyle=zigzag,labeloffset=-1.3](D)(E){$R_2=1k$}
\capacitor[dipolestyle=chemical,
           labeloffset=-1,
           tension,
           tensionlabel=$v_D\left(t\right)$,
           tensionlabeloffset=1.5](F)(G){$C$}
\wire(A)(E)
\wire(E)(G)
\newground(E)
\end{pspicture}
\caption{\label{fig:descargcond}Circuito de descarregamento do condensador.}
\end{figure}

\begin{table}[h]
\centering
\caption{\label{tab:vcondensador}Tens\~{a}o ($v_D$) nos terminais do condensador em fun\c{c}\~{a}o do instante de tempo ($t$).}
\begin{tabular}{|c|c|c|c|c|c|c|c|c|c|c|}
\hline 
$t\pm 1$ (s) & 0 & 6 & 12 & 18 & 24 & 30 & 36 & 42 & 48 & 54\\ 
\hline 
$v_D\pm 0.01$ (V) & 5.95 & 3.29 & 1.81 & 0.99 & 0.54 & 0.29 & 0.16 & 0.09 & 0.04 & 0.03\\ 
\hline 
\end{tabular} 
\end{table}

\begin{parts}
\part[25]
Sabendo que a d.d.p. nos terminais do condensador decresce exponencialmente, segundo a Eq.~\ref{eq:vdescarga}, linearize esta express\~{a}o e determine experimentalmente a constante de tempo de descarga do condensador $\tau$ e respetivo erro associado $\Delta\tau$, usando as medidas experimentais diretas presentes na Tabela~\ref{tab:vcondensador}.

Nota: N\~{a}o precisa de representar graficamente os pontos experimentais e a reta da lineariza\c{c}\~{a}o.

\begin{equation}
\label{eq:vdescarga}
v_D\left(t\right)=\varepsilon\left[e^{\left(-\frac{t}{\tau}\right)}\right]
\end{equation}

\part[15]
Explique qual o significado f\'{i}sico da constante de tempo de descarregamento do condensador $\tau$. Pode recorrer a esbo\c{c}os gr\'{a}ficos, caso lhe seja conveniente.

\begin{solution}

\begin{pspicture}[showgrid=false](-1.5,-0.5)(9,3.5)
\psset{xunit=0.3cm,yunit=0.3cm}
\psplot[plotpoints=10,linecolor=blue]{0}{6}{10 10 x mul 6 div sub}
\psline[linecolor=blue,linestyle=dashed,dash=3pt 2pt](6,3.7)(6,0)
\psplot[linecolor=red,plotpoints=100]{0}{30}{10 Euler x 6 div neg exp mul}
\psline{->}(0,0)(0,12)
\psline{->}(0,0)(30,0)
\psline[linecolor=red,linestyle=dashed,dash=3pt 2pt](0,3.7)(6,3.7)
\rput[tl](30,0){$t$}
\rput[br](0,12){$v_D$}
\rput[t](6,-0.5){$\tau$}
\rput[r](-0.5,10){$\varepsilon$}
\rput[r](-0.5,3.7){$\frac{\varepsilon}{e}$}
\rput[r](-0.5,2.2){$\approx37\%~\varepsilon$}
\end{pspicture}

\end{solution}

\part[10]
De que modo teria de alterar o circuito da Fig.~\ref{fig:descargcond} para aumentar o valor da tens\~{a}o inicial do condensador $v_D(t=0)$?
\end{parts}