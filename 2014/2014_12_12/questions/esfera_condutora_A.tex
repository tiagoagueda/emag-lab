%
%

\question
\textbf{Trabalho Pr\'{a}tico 1 : Eletrost\'{a}tica}

Recorde o Trabalho Pr\'{a}tico 1, no qual determinou a capacidade el\'{e}trica de uma esfera condutora.

Neste trabalho aplicou diversos potenciais elevados a uma esfera condutora de raio $R\pm\Delta R=4.50\pm0.05~\text{cm}$, adquirindo esta uma carga. Os valores dessa carga $Q$, para diferentes potenciais aplicados $V$, est\~{a}o registados na Tabela~\ref{tab:qesfera}.

\begin{table}[h]
\centering
\caption{\label{tab:qesfera}Carga ($Q$) da esfera em fun\c{c}\~{a}o do potencial ($V$) aplicado.}
\begin{tabular}{|c|c|c|c|c|c|c|}
\hline 
$V\pm 0.01$ (kV) & 0.52 & 1.85 & 3.14 & 3.93 & 5.39 & 6.43 \\ 
\hline 
$Q\pm 1$ (nC) & 12 & 17 & 25 & 29 & 35 & 41 \\ 
\hline 
\end{tabular} 
\end{table}

\begin{parts}
\part[15]
A carga da esfera n\~{a}o \'{e} medida diretamente. Explique sucintamente, como se determina o valor da carga.

\part[25]
Determine o valor experimental da capacidade el\'{e}trica da esfera $C_\text{esfera}$ e respectivo erro. (Nota: N\~{a}o precisa de representar graficamente os valores experimentais bem como a reta da lineariza\c{c}\~{a}o.)

\part[10]
Compare o valor experimental com o valor esperado, sabendo que a tens\~{a}o aplicada \`{a} esfera condutora \'{e} proporcional \`{a} carga $Q$ de acordo com a Eq.~\ref{eq:vpropq}. Comente o resultado.

\begin{equation}
\label{eq:vpropq}
V=\frac{1}{4\pi\varepsilon_0 R}Q
\end{equation}
\end{parts}